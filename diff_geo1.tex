\documentclass[a4j,dvipdfmx]{jsarticle}
\usepackage{amsmath,amssymb}
\usepackage{siunitx}
\usepackage{ascmac}
\usepackage{amsthm}
\usepackage{bm}

\newcommand{\rank}{\mathrm{rank}}
% \usepackage[dvipdfmx]{hyperref}

\title{二時間で学ぶRiemann幾何}

\begin{document}
    \maketitle

    \section{復習など}
        詳しくは以前送ったベクトル解析のノート参照. ここでは必要な最低限の要点のみ復習する. まずは基本的な用語についてザックリ復習する.

        \begin{description}
            \item[(幾何)ベクトル] 単純に言えば, 向きと方向を持つ量. 通常矢印で表現される.
            \item[関数] あるものに対してある実数を対応させる写像.
            \item[曲面片] 3次元Euclid空間上で, 二つのパラメータをもつ位置ベクトル関数があったとき, その関数の各成分から作られるJacobi行列の階数が2ならば, この位置ベクトルが表す点全体を曲面片という.
            \item[曲面] 曲面片の和集合(可算和でなくてよい.)
        \end{description}

        さて, ベクトル解析での内容と重複するが, 簡単に曲面論の復習をしてみよう. 我々は3次元Euclid空間上にある曲面について, その性質を調べていたのだった.
        まず, 曲面(片)は
        \begin{equation}
            \bm{p}(u,v) = (x(u,v), y(u,v), z(u,v))
        \end{equation}
        なるようにかける. (本来なら列ベクトルにするのが一般的だが, ここではこだわらない.) ここで, $u,v$がパラメータで, $x,y,z$が三階連続微分可能な関数である.
        先ほどの曲面片となるための条件を式でかけば
        \begin{equation}
            \rank \begin{bmatrix}
                x_u & y_u & z_u \\
                x_v & y_v & z_v
            \end{bmatrix} = 2
        \end{equation}
        ということになる. これはつまり, 曲面の接ベクトルが互いに一次独立であるということで, これにより接平面が張れることが保証されている.

        以下は, 簡単な復習のために少し厳密性を犠牲にしていることを断っておく. まず曲面上の十分近い二点の微小な変位ベクトル$d\bm{p}$は
        \begin{equation}
            d\bm{p} = \bm{p}_u du + \bm{p}_v dv
        \end{equation}
        として与えられる. これは多変数関数の全微分と同じ形なので理解しやすいだろう. 直感的には, uv平面上でu軸方向に$du$だけ, v軸方向に$dv$
        だけ進んだ場合, 曲面上のu曲線, v曲線の接線方向にそれぞれ$\bm{p}_udu, \bm{p}_vdv$だけ増加すると想像するとわかりやすい.
        ゆえに曲面上の微小距離$ds^2$は
        \begin{equation}
            ds^2 = Edu^2+2Fdudv+Gdv^2
        \end{equation}
        として与えられる. ここで$E=\bm{p}_u^2, F=\bm{p}_u\cdot\bm{p}_v, G=\bm{p}_v^2$である. これを曲面の\textbf{第一基本形式}という.

        さて, 次に, 曲面上の法線ベクトルを考える. それは次のように与えられる.
        \begin{equation}
            \bm{e}=\frac{\bm{p}_u\times\bm{p}_v}{|\bm{p}_u\times\bm{p}_v|}
        \end{equation}
        これより, 曲面の\textbf{第二基本形式}は以下で与えられる.
        \begin{equation}
            \mathrm{II} = -d\bm{e}\cdot d\bm{p}
        \end{equation}
        簡単な式変形から, これは次のように書ける.
        \begin{equation}
            \mathrm{II}=Ldu^2+2Mdudv+Ndv^2
        \end{equation}
        ただし, $L=\bm{e}\cdot \bm{p}_{uu},M=\bm{e}\cdot \bm{p}_{uv}, N=\bm{e}\cdot \bm{p}_{vv}$である. この二次形式の対称行列はある種Hesse行列に対応しているから,
        Hessianを計算すれば, 曲面が凸かどうかがわかる.

        曲面の第一基本形式と第二基本形式を用いれば, 次のように\textbf{Gauss曲率}が定義できる.
        \begin{eqnarray}
            K=\frac{LN-M^2}{EG-F^2}
        \end{eqnarray}
        実はこのGauss曲率は内在的な量, つまり第一基本形式のみから決まる.(Theorema egregium, 最も素晴らしい定理) 
    \clearpage
    \section{曲面上のRiemann計量}
\end{document}