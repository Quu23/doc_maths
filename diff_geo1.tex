\documentclass[a4j,dvipdfmx]{jsarticle}
\usepackage{amsmath,amssymb}
\usepackage{siunitx}
\usepackage{ascmac}
\usepackage{amsthm}
\usepackage{bm}
\usepackage[dvipdfmx]{hyperref}

\usepackage[dvipdfmx]{graphicx}
\usepackage{graphics}

\newcommand{\rank}{\mathrm{rank}}


\title{二時間で学ぶRiemann幾何の初歩}
\author{理学同好会}

\begin{document}
    \maketitle

    \section{復習など}
        詳しくは以前送ったベクトル解析のノート参照. ここでは必要な最低限の要点のみ復習する. まずは基本的な用語についてザックリ復習する.

        \begin{description}
            \item[(幾何)ベクトル] 単純に言えば, 向きと方向を持つ量. 通常矢印で表現される.
            \item[関数] あるものに対してある実数を対応させる写像.
            \item[曲面片] 3次元Euclid空間上で, 二つのパラメータをもつ位置ベクトル関数があったとき, その関数の各成分から作られるJacobi行列の階数が2ならば, この位置ベクトルが表す点全体を曲面片という.
            \item[曲面] 曲面片の和集合(可算和でなくてよい.)
        \end{description}

        さて, ベクトル解析での内容と重複するが, 簡単に曲面論の復習をしてみよう. 我々は3次元Euclid空間上にある曲面について, その性質を調べていたのだった.
        まず, 曲面(片)は
        \begin{equation}
            \bm{p}(u,v) = (x(u,v), y(u,v), z(u,v))
        \end{equation}
        なるようにかける. (本来なら列ベクトルにするのが一般的だが, ここではこだわらない.) ここで, $u,v$がパラメータで, $x,y,z$が三階連続微分可能な関数である.
        先ほどの曲面片となるための条件を式でかけば
        \begin{equation}
            \rank \begin{bmatrix}
                x_u & y_u & z_u \\
                x_v & y_v & z_v
            \end{bmatrix} = 2
        \end{equation}
        ということになる. これはつまり, 曲面の接ベクトルが互いに一次独立であるということで, これにより接平面が張れることが保証されている.

        以下は, 簡単な復習のために少し厳密性を犠牲にしていることを断っておく. まず曲面上の十分近い二点の微小な変位ベクトル$d\bm{p}$は
        \begin{equation}
            d\bm{p} = \bm{p}_u du + \bm{p}_v dv
        \end{equation}
        として与えられる. これは多変数関数の全微分と同じ形なので理解しやすいだろう. 直感的には, uv平面上でu軸方向に$du$だけ, v軸方向に$dv$
        だけ進んだ場合, 曲面上のu曲線, v曲線の接線方向にそれぞれ$\bm{p}_udu, \bm{p}_vdv$だけ増加すると想像するとわかりやすい.
        ゆえに曲面上の微小距離$ds^2$は
        \begin{equation}
            ds^2 = Edu^2+2Fdudv+Gdv^2
        \end{equation}
        として与えられる. ここで$E=\bm{p}_u^2, F=\bm{p}_u\cdot\bm{p}_v, G=\bm{p}_v^2$である. これを曲面の\textbf{第一基本形式}という.

        さて, 次に, 曲面上の法線ベクトルを考える. それは次のように与えられる.
        \begin{equation}
            \bm{e}=\frac{\bm{p}_u\times\bm{p}_v}{|\bm{p}_u\times\bm{p}_v|}
        \end{equation}
        これより, 曲面の\textbf{第二基本形式}は以下で与えられる.
        \begin{equation}
            \mathrm{II} = -d\bm{e}\cdot d\bm{p}
        \end{equation}
        簡単な式変形から, これは次のように書ける.
        \begin{equation}
            \mathrm{II}=Ldu^2+2Mdudv+Ndv^2
        \end{equation}
        ただし, $L=\bm{e}\cdot \bm{p}_{uu},M=\bm{e}\cdot \bm{p}_{uv}, N=\bm{e}\cdot \bm{p}_{vv}$である. この二次形式の対称行列はある種Hesse行列に対応しているから,
        Hessianを計算すれば, 曲面が凸かどうかがわかる.

        曲面の第一基本形式と第二基本形式を用いれば, 次のように\textbf{Gauss曲率}が定義できる.
        \begin{eqnarray}
            K=\frac{LN-M^2}{EG-F^2}
        \end{eqnarray}
        実はこのGauss曲率は内在的な量, つまり第一基本形式のみから決まる.(Theorema egregium, 最も素晴らしい定理) 
    \clearpage
    \section{曲面上のRiemann計量}
        これまでの議論は, あくまでEuclid空間上に曲面があって, そのうえで解析していた. つまり``曲面の外''がある場合であった.
        そこで次に考えたいのは, この``外''の空間をとっぱらって, 曲面そのものだけを見たときに, 曲面の性質をどう調べるかということである.
        つまり, 曲面がEuclid空間上にあることをいったん忘れて, 曲面そのものを考えたい. しかし, 曲面の外がないということは曲面を位置ベクトルで書けないということであって, これまでのやり方が
        あまり通用しなそうにも思える. なぜなら, 位置ベクトルはEuclid空間に付随しているものであって, 曲面そのものに付随しているわけではないからである.

        逆に, 曲面に付随しているものとして考えると, まず思いつくのは接ベクトルがある. 接ベクトルは曲面に張り付いているので, 外の空間がなくても定義できそうである. 
        そこで思い切って曲面の代わりに, 以下の第一基本形式が与えられたとして, 曲面の性質を調べることにしよう.
        \begin{equation}
            ds^2 = Edu^2+2Fdudv+Gdv^2 \label{eq:Riemann metric}
        \end{equation}
        ただし, $u,v$はあるパラメータで, 第一基本形式はuv平面上のある領域$D$内で定義されているものとする. また, 第一基本形式は(二次形式の意味で)正値形式であるとする. このようにして与えられた$ds^2$を
        領域$D$上の\textbf{Riemann計量}(Riemann metric)という. なお
        \begin{equation}
            g_{ij}=\begin{bmatrix}
                E & F \\ F & G
            \end{bmatrix}
        \end{equation}
        とすれば, $ds^2 = g_{ij}du^{i}du^{j}$とも書ける. ただし, $u^1=u,u^2=v$であり, 上下で同じ添え字が出た場合は, 1から2までの和をとるものとする.(Einsteinの規約) これは後からまた示すように, ``接ベクトル''の
        内積を定義したことになる. 先ほどの位置ベクトルを使った議論における$E,F,G$の定義をみればなんとなくわかる.
        

        しかし, このままでは何を考えていいのかよくわからない. $u,v$というパラメータが曲面の何を表しているかが不明瞭だし, そもそも, 式中の$du,dv$の記号をはっきり定義しないまま使ってきているのがよくない. そこでまずこれらの正体をはっきりと示そう.
    \clearpage
    \section{接ベクトル場}
        $du,dv$の正体を探るには, まず接ベクトルについて改めて定式化する必要がある. そのための手がかりとして, 曲面上の曲線を考え, その速度ベクトルを考えてみることにする.

        さらりと``曲線''と言ってしまったが, まず曲面上の曲線とはどのようなものなのだろうか. まずは曲面上の曲線について考えてみよう. 曲面を$M$として, 次の写像を考えてみる.
        \begin{equation}
            c:[-\varepsilon,\varepsilon]\to M \quad (\varepsilon\in \mathrm{R}, \varepsilon > 0)
        \end{equation}
        これは, パラメータ($t$とかいておく)を一つ決めると, 曲面$M$上の一点が決まる写像である. その一点を$c(t)$と書くことにする. これだけだと, $t$の値に対してとびとびの点$c(t)$を取りうる可能性も
        ある(むしろそれが一般的)であるが, ここでは常識的な, 点が``連続的''に変化しているものとする. そうすると, $c$の像は連続的な点の集まりだから曲線といえなくもない. 

        ではつぎにこの曲線$c$の速度ベクトルなるものを考えたい. 速度といえば微分であるから, この曲線を微分することを考えたい. しかし, 一般の写像の場合の微分は定義していない. 実際微分の定義を形式的に当てはめようとすると
        \begin{equation*}
            \lim_{\Delta t\to 0} \frac{c(t+\Delta t)-c(t)}{\Delta t}
        \end{equation*}
        を計算することになるが, 分子は$M$の点であって, これらの加法は一般に定義されていないのだから, 微分しようにもundefinedというわけである.

        そこで, $M$の点を数としてみなせれば, 微分が計算できて都合がよいということがわかる. 曲面上の点を数値で表すということは, 曲面に座標を導入するということであるが, 最初に述べた通り
        今はあくまで曲面はEuclid空間にはおかない前提であるから, 通常の座標軸のように設定するわけにもいかない. そこで, 曲面の外側で座標軸を設置するのではなく, 曲面の表面に座標を張ればよいということがわかる.
        しかし, 現実の曲面を想像すればわかるように, 曲面上に座標格子を広げようにも, それだけで曲面全体を覆うことはできないし, 逆に軸同士などがぶつかってしまうこともある. 例えば球であれば, 球面上のある点から座標軸を伸ばしていくと一周して元の場所に戻ってしまい
        軸それ自身とぶつかってしまう. しかし, 曲面上のある点と, そのごくごく小さいその周囲(これを\textbf{近傍}という)だけであれば, 座標軸がぶつかることはないだろう. また, 曲面上の各点に対してその近傍を考えることで, 近傍全体の和集合が曲面を包んでいることもわかる.

        これを数学的に定義するには位相空間に関する知識がいるから深追いしない. 今では単に曲面の局所的な部分で座標を張っていると考えればよい. これを局所座標という. なお, このような座標が張れる集合を\textbf{多様体}という. 曲面はこの多様体のうち, 座標軸が二本の二次元多様体と呼ばれるものにあたるのである.
        \begin{figure}[h]
            \centering
            \includegraphics[scale=0.5]{img/diff_geo1/surface_axis.png}
            \caption{局所座標}
        \end{figure}
        
        さて, 局所座標を定義したことにより曲線の速度ベクトルが定義できるようになった. まず, $M$の点$p$の局所座標を$(u,v)$とかくことにしよう. つまり, $p\in M$を$p=(u,v)$のように書くのであって, これが式\eqref{eq:Riemann metric}中の$u,v$
        の正体である. これより, 曲線は$c(t)=(u(t),v(t))$と書ける. これより, $t=0$における速度ベクトルは
        \begin{equation}
            \left.\frac{dc}{dt}\right|_{t=0}=\left(\frac{du}{dt}(0),\frac{dv}{dt}(0)\right)
        \end{equation}
        として定義される...と言いたいところだが, 実はそうではない. こう定義してしまうと, 局所座標系$(u,v)$の取り方によって速度ベクトルが変わってしまうのである. 速度ベクトルないしこれから定義する接ベクトルは, 曲面上の\underline{点のみ}に依存するのであって, 局所座標系にまで依存されると困ってしまう.
        
        しかし, 微分を使うというアイデア自体は間違っていないはずだから, ここも思い切って, 次のように考えてみることにしよう.
        \begin{itembox}[l]{速度ベクトルの定義}
            曲線$c$の$t=0$における\textbf{速度ベクトル}$\bm{v}_c$は, 曲線$c$に沿った\textbf{方向微分そのもの}と考える. すなわち, ある点$p=c(0)$周りで定義された(微分可能性などを持っている)任意の関数$f$に対して
            \begin{equation}
                \bm{v}_c(f) = \left.\frac{df(c(t))}{dt}\right|_{t=0}
            \end{equation}
            を返す関数$\bm{v}_c:\{\text{関数}\}\to \mathbb{R}$である.
        \end{itembox}
        さらに, 方向微分そのものをより抽象的にしよう. すなわち, 方向微分とは微分演算が持つ「線形性」と「Leibniz則」のみを仮定した, 
        点$p$周りで定義された関数から実数への関数と考えるのである. そのようにすることで, ある点$p$における方向微分全体がベクトル空間となる.(もちろん方向微分同士の加法などは, 自然な意味での加法ととる)

        点$p$における方向微分全体を$D_p(M)$と書くことにする. しかし, $D_p(M)$を定義しておいてはなんだが, 方向微分全体はあまりにも範囲が広すぎる. 実際に我々が必要なのは接ベクトルであったから, 
        方向微分全体の部分空間だけ知っていれば十分であろう. では, 接ベクトル場はどのように定義するだろうか. 通常の曲面論から類推すると, これはu軸, v軸に沿った速度ベクトルであるとわかる. 
        座標軸は二本あるから, 接ベクトルとして(線形独立なものは)二つとれるだろう. u軸, v軸に沿った点$p$における接ベクトルをそれぞれ
        \begin{equation}
            \left(\frac{\partial}{\partial u}\right)_p,\quad \left(\frac{\partial}{\partial v}\right)_p
        \end{equation}
        とかこう. この二つのベクトルによって張られる$D_p(M)$の部分空間を$T_p(M)$と書き, 点$p$における\textbf{接ベクトル空間}という.
        簡単に言えば, この接ベクトル空間が曲面$M$の接平面にあたる. $T_p(M)$の元を$p$における接ベクトルという. そして, $M$上の各点でそれぞれ(局所座標が通じる限り)u軸, v軸に沿った接ベクトルが定義できるから,
        それらを集めて$\displaystyle \frac{\partial}{\partial u} = \left\{\left(\frac{\partial}{\partial u}\right)_p\right\}_{p\in U}$のようにかく. $U$は$M$上で局所座標$(u,v)$が張られている範囲.
        そしてこれを\textbf{接ベクトル場}という. 以下簡単に書くため, 接ベクトル場をそれぞれ$\partial_u, \partial_v$とかくこともある.

        なお, 速度ベクトル$\bm{v}_c$は接ベクトル, つまり, $\displaystyle\left(\frac{\partial}{\partial u}\right)_p, \left(\frac{\partial}{\partial v}\right)_p$の線形結合で書ける. これは連鎖律の公式を使えばよい.
    \clearpage
    \section{微分形式}
        あああ
    \section{曲面の構造方程式}
        いいい
        

\end{document}